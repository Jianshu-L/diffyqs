\chapter{Eigenvalue problems} \label{SL:chapter}

%%%%%%%%%%%%%%%%%%%%%%%%%%%%%%%%%%%%%%%%%%%%%%%%%%%%%%%%%%%%%%%%%%%%%%%%%%%%%%

\section{Sturm-Liouville problems}
\label{slproblems:section}

\sectionnotes{2 lectures\EPref{, \S10.1 in \cite{EP}}\BDref{,
\S11.2 in \cite{BD}}}

\subsection{Boundary value problems}

We have encountered several different eigenvalue problems such as:
\begin{equation*}
X''(x) + \lambda X(x) = 0
\end{equation*}
with different boundary
conditions\index{Dirichlet boundary conditions}%
\index{Neumann boundary conditions}%
\index{Mixed boundary conditions}
\begin{equation*}
\begin{array}{rrl}
X(0) = 0 & ~~X(L) = 0 & ~~\text{(Dirichlet) or}, \\
X'(0) = 0 & ~~X'(L) = 0 & ~~\text{(Neumann) or}, \\
X'(0) = 0 & ~~X(L) = 0 & ~~\text{(Mixed) or}, \\
X(0) = 0 & ~~X'(L) = 0 & ~~\text{(Mixed)}, \ldots
\end{array}
\end{equation*}
For example for the insulated wire,
Dirichlet conditions correspond to applying a
zero temperature at the ends, Neumann means insulating the ends, etc\ldots.
Other types of endpoint conditions also arise naturally, such as
the \emph{\myindex{Robin boundary conditions}}
\begin{equation*}
hX(0) - X'(0) = 0 \qquad hX(L) + X'(L) = 0 ,
\end{equation*}
for some constant $h$.  These conditions come up when the ends are immersed
in some medium.

Boundary problems came up in the study of the heat equation $u_t =
k u_{xx}$ when we were trying to solve the equation by the method of
separation of
variables.  In the computation
we encountered a certain eigenvalue problem and found the
eigenfunctions
$X_n(x)$.  We then found the \emph{\myindex{eigenfunction decomposition}} of
the initial
temperature $f(x) = u(x,0)$ in terms of the eigenfunctions
\begin{equation*}
f(x) = \sum_{n=1}^\infty c_n X_n(x) .
\end{equation*}
Once we had this decomposition
and found suitable $T_n(t)$ such that $T_n(0) = 1$
and $T_n(t)X(x)$ were solutions,
the solution to the original problem including the initial
condition could be written as
\begin{equation*}
u(x,t) = \sum_{n=1}^\infty c_n T_n(t) X_n(x) .
\end{equation*}

\medskip

We will try to solve more general problems using this method.
First,
we will study 
second order linear equations of the form
\begin{equation} \label{SL:eq}
\frac{d}{dx} \left( p(x) \frac{dy}{dx} \right)
- q(x) y + \lambda r(x) y = 0 .
\end{equation}
Essentially
any second order linear equation
of the form $a(x) y'' + b(x) y' + c(x) y + \lambda d(x) y = 0$
can be written as \eqref{SL:eq}
after multiplying by a proper factor.

\begin{example}[Bessel]
Put the following equation into the form \eqref{SL:eq}:
\begin{equation*}
x^2 y'' + xy' + \left(\lambda x^2 - n^2\right)y = 0 .
\end{equation*}
Multiply both sides by $\frac{1}{x}$ to obtain
\begin{equation*}
\frac{1}{x} \left( x^2 y'' + xy' + \left(\lambda x^2 - n^2\right)y \right)
=
x y'' + y' + \left(\lambda x - \frac{n^2}{x}\right)y 
=
\frac{d}{dx} \left( x \frac{dy}{dx} \right)
- \frac{n^2}{x} y + \lambda x y  = 0.
\end{equation*}
\end{example}

The so-called 
\emph{\myindex{Sturm-Liouville problem}}%
\footnote{Named after the French mathematicians
\href{https://en.wikipedia.org/wiki/Jacques_Charles_Fran\%C3\%A7ois_Sturm}{Jacques Charles Fran\c{c}ois Sturm}
(1803--1855) and
\href{https://en.wikipedia.org/wiki/Liouville}{Joseph Liouville}
(1809--1882).} is to seek
nontrivial solutions to
\begin{equation} \label{sl:slprob}
\boxed{~~
\begin{aligned}
&\frac{d}{dx} \left( p(x) \frac{dy}{dx} \right)
- q(x) y + \lambda r(x) y = 0, \qquad a < x < b, \\
&\alpha_1 y(a) - \alpha_2 y'(a) = 0, \\
&\beta_1 y(b) + \beta_2 y'(b) = 0.
\end{aligned}
~~}
\end{equation}
In particular, we seek $\lambda$s that allow for nontrivial solutions.
The $\lambda$s that admit nontrivial solutions
are called the \emph{eigenvalues\index{eigenvalue}}
and the corresponding
nontrivial solutions are called
\emph{eigenfunctions\index{eigenfunction}}.
The constants $\alpha_1$ and $\alpha_2$ should not be both zero, same for
$\beta_1$ and $\beta_2$.

\begin{theorem} \label{sl:slregthm}
Suppose $p(x)$, $p'(x)$, $q(x)$ and $r(x)$ are continuous on $[a,b]$
and suppose $p(x) > 0$ and $r(x) > 0$ for all $x$ in $[a,b]$.
Then the Sturm-Liouville problem \eqref{sl:slprob}
has an increasing sequence of eigenvalues
\begin{equation*}
\lambda_1 < \lambda_2 < \lambda_3 < \cdots 
\end{equation*}
such that
\begin{equation*}
\lim_{n \to \infty} \lambda_n = +\infty
\end{equation*}
and such that to each $\lambda_n$ there is (up to a constant multiple)
a single eigenfunction $y_n(x)$. 

Moreover, if $q(x) \geq 0$ and $\alpha_1, \alpha_2, \beta_1, \beta_2 \geq 0$,
then $\lambda_n \geq 0$ for all $n$.
\end{theorem}

Problems satisfying the hypothesis of the
theorem are called
\emph{regular Sturm-Liouville problems\index{regular Sturm-Liouville problem}}%
\index{Sturm-Liouville problem!regular}
and we will only consider such problems here.
That is, a regular problem is one where
$p(x)$, $p'(x)$, $q(x)$ and $r(x)$ are continuous, $p(x) > 0$, $r(x) > 0$,
$q(x) \geq 0$, and $\alpha_1, \alpha_2, \beta_1, \beta_2 \geq 0$.
Note: Be careful about the signs.  Also be careful about the inequalities
for $r$ and $p$, they
must be strict for all $x$!

When zero
is an eigenvalue, we usually
start labeling the eigenvalues at 0 rather than at 1 for convenience.

\begin{example}
The problem $y''+\lambda y$, $0 < x < L$, $y(0) = 0$, and $y(L) = 0$
is a regular Sturm-Liouville problem.  $p(x) = 1$, $q(x) = 0$, $r(x) = 1$,
and we have $p(x) = 1 > 0$ and $r(x) = 1 > 0$.
The eigenvalues are $\lambda_n = \frac{n^2 \pi^2}{L^2}$ and eigenfunctions
are $y_n(x) = \sin(\frac{n\pi}{L} x)$.  All eigenvalues are nonnegative as
predicted by the theorem.
\end{example}

\begin{exercise}
Find eigenvalues and eigenfunctions for
\begin{equation*}
y'' + \lambda y = 0, \quad y'(0) = 0, \quad y'(1) = 0.
\end{equation*}
Identify
the $p, q, r, \alpha_j, \beta_j$.
Can you use the theorem to make the search for eigenvalues easier?
(Hint: Consider the condition $-y'(0)=0$)
\end{exercise}

\begin{example}
Find eigenvalues and eigenfunctions of the problem
\begin{align*}
& y''+\lambda y = 0, \quad 0 < x < 1 , \\
& hy(0)- y'(0) = 0, \quad y'(1)  = 0, \quad h > 0.
\end{align*}

These equations give a regular Sturm-Liouville problem.

\begin{exercise}
Identify $p, q, r, \alpha_j, \beta_j$ in the example above.
\end{exercise}

First note that
$\lambda \geq 0$ by \thmref{sl:slregthm}.
Therefore, the general solution (without boundary conditions) is
\begin{equation*}
\begin{aligned}
 & y(x) = A \cos (\! \sqrt{\lambda}\, x) + B \sin (\!
\sqrt{\lambda}\, x) & & \qquad \text{if } \; \lambda > 0 , \\
& y(x) = A x + B & & \qquad \text{if } \; \lambda = 0 .
\end{aligned}
\end{equation*}

Let us see if $\lambda = 0$ is an eigenvalue:
We must satisfy $0 = hB - A$ and $A = 0$, hence $B=0$ (as $h > 0$).
Therefore, 0 is not an eigenvalue (no nonzero solution, so no eigenfunction).

Now let us
try $\lambda > 0$.  We plug in the boundary conditions.
\begin{align*}
& 0 = h A - \sqrt{\lambda}\, B , \\
& 0 = -A \sqrt{\lambda}\, \sin (\!\sqrt{\lambda}) +B \sqrt{\lambda}\,
\cos (\!\sqrt{\lambda}) .
\end{align*}
If $A=0$, then $B=0$ and vice-versa, hence both are nonzero.
So $B = \frac{hA}{\sqrt{\lambda}}$, and
$0 = -A \sqrt{\lambda}\, \sin (\! \sqrt{\lambda}) + \frac{hA}{\sqrt{\lambda}}
\sqrt{\lambda}\, \cos (\! \sqrt{\lambda})$.  As $A \not= 0$ we get
\begin{equation*}
0 = 
- \sqrt{\lambda}\, \sin (\! \sqrt{\lambda}) + h \cos (\! \sqrt{\lambda}) ,
\end{equation*}
or
\begin{equation*}
\frac{h}{\sqrt{\lambda}} = \tan \sqrt{\lambda} .
\end{equation*}

Now use a computer to find $\lambda_n$.  There are tables available,
though using a computer or a graphing calculator
is far more convenient nowadays.
Easiest method is to plot the functions 
$\nicefrac{h}{x}$ and $\tan x$ and see for which $x$ they intersect.
There is an infinite number of intersections.  Denote
the first intersection
by $\sqrt{\lambda_1}$,
the second intersection by $\sqrt{\lambda_2}$, etc\ldots.
For example, when
$h=1$, we get $\sqrt{\lambda_1} \approx 0.86$, 
$\sqrt{\lambda_2} \approx 3.43$, \ldots.
That is $\lambda_1 \approx 0.74$, $\lambda_2 \approx 11.73$, \ldots.
A plot for $h=1$ is given in \figurevref{sl:tanx1overxfig}.
The appropriate eigenfunction
(let $A = 1$ for convenience, then
$B=\nicefrac{h}{\sqrt{\lambda}}$) is
\begin{equation*}
y_n(x) = \cos (\! \sqrt{\lambda_n}\, x ) + \frac{h}{\sqrt{\lambda_n}} \,
\sin (\!\sqrt{\lambda_n} \, x ) .
\end{equation*}
When $h=1$ we get (approximately)
\begin{equation*}
y_1(x) \approx \cos (0.86\, x ) + \frac{1}{0.86} \,
\sin (0.86 \, x ) , \qquad
y_2(x) \approx \cos (3.43\, x ) + \frac{1}{3.43} \,
\sin (3.43 \, x ) , \qquad \ldots .
\end{equation*}
\begin{figure}[h!t]
\capstart
\begin{center}
\diffyincludegraphics{width=3in}{width=4.5in}{sl-tanx1overx}
\caption{Plot of $\frac{1}{x}$ and $\tan x$.%
\label{sl:tanx1overxfig}}
\end{center}
\end{figure}
\end{example}

\subsection{Orthogonality}

We have seen the notion of orthogonality before.  For example,
we have shown that $\sin (nx)$ are orthogonal for distinct $n$ on $[0,\pi]$.
For general Sturm-Liouville problems we will need a more general setup.
Let $r(x)$
be a \emph{\myindex{weight function}} (any function, though generally we will
assume it is positive) on $[a,b]$.  Two functions $f(x)$, $g(x)$
are said to be
\emph{orthogonal\index{orthogonal!with respect to a weight}} 
with respect to the weight function
$r(x)$ when
\begin{equation*}
\int_a^b f(x) \, g(x) \, r(x) ~dx = 0 .
\end{equation*}
In this setting,
we define the \emph{inner product\index{inner product of functions}} as
\begin{equation*}
\langle f , g \rangle \overset{\text{def}}{=} \int_a^b f(x) \, g(x) \, r(x) ~dx ,
\end{equation*}
and then say $f$ and $g$ are orthogonal whenever $\langle f , g \rangle = 0$.
The results and concepts are again analogous to 
finite dimensional linear algebra.

The idea of the given inner product is that those $x$ where $r(x)$
is greater have more weight.
Nontrivial (nonconstant)
$r(x)$ arise naturally, for example from a change of variables.
Hence, you could
think of a change of variables such that $d\xi = r(x)~ dx$.

We have the following orthogonality property of eigenfunctions of a
regular Sturm-Liouville problem.

\begin{theorem}
Suppose we have a regular Sturm-Liouville problem
\begin{align*}
&\frac{d}{dx} \left( p(x) \frac{dy}{dx} \right)
- q(x) y + \lambda r(x) y = 0 , \\
&\alpha_1 y(a) - \alpha_2 y'(a) = 0 , \\
&\beta_1 y(b) + \beta_2 y'(b) = 0 .
\end{align*}
Let $y_j$ and $y_k$ be two distinct eigenfunctions for two
distinct eigenvalues $\lambda_j$ and $\lambda_k$.  Then
\begin{equation*}
\int_a^b y_j(x) \, y_k(x) \, r(x) ~dx = 0,
\end{equation*}
that is, $y_j$ and $y_k$ are orthogonal with respect to the weight function
$r$.
\end{theorem}

Proof is very similar to the analogous theorem from \sectionref{bvp:section}.
It can also be found in many books including, for example,
Edwards and Penney \cite{EP}.

\subsection{Fredholm alternative}

We also have the \emph{Fredholm alternative} theorem we talked about before
for all regular Sturm-Liouville problems.  We state it here for completeness.

\begin{theorem}[Fredholm alternative]%
\index{Fredholm alternative!Sturm-Liouville problems}
Suppose that we have a regular Sturm-Liouville problem.
Then either
\begin{align*}
&\frac{d}{dx} \left( p(x) \frac{dy}{dx} \right)
- q(x) y + \lambda r(x) y = 0 , \\
&\alpha_1 y(a) - \alpha_2 y'(a) = 0 , \\
&\beta_1 y(b) + \beta_2 y'(b) = 0 ,
\end{align*}
has a nonzero solution, or
\begin{align*}
&\frac{d}{dx} \left( p(x) \frac{dy}{dx} \right)
- q(x) y + \lambda r(x) y = f(x) , \\
&\alpha_1 y(a) - \alpha_2 y'(a) = 0 , \\
&\beta_1 y(b) + \beta_2 y'(b) = 0 ,
\end{align*}
has a unique solution for any $f(x)$ continuous on $[a,b]$.
\end{theorem}

This theorem is used in much the same way as we did before in
\sectionref{sec:scs}.  It is used when solving more general nonhomogeneous
boundary value problems.  The theorem does not help us solve the problem, but
it tells us when a unique solution exists, so
that we know when to spend time looking for it.  To solve the problem
we decompose $f(x)$ and $y(x)$ in terms of eigenfunctions of the
homogeneous
problem, and then solve for the coefficients of the series for $y(x)$.

\subsection{Eigenfunction series}

What we want to do with the eigenfunctions once we have them is to
compute the \emph{\myindex{eigenfunction decomposition}} of an arbitrary
function $f(x)$.  That is, we wish to write
\begin{equation} \label{sl:fdecomp}
f(x) = \sum_{n=1}^\infty c_n y_n(x) ,
\end{equation}
where $y_n(x)$ are eigenfunctions.
We wish to find out if we can represent
any function $f(x)$ in this way,
and if so, we wish to calculate $c_n$ (and of course we would want to know if
the sum converges).  OK, so imagine
we could write $f(x)$ as \eqref{sl:fdecomp}.  We will assume convergence and
the ability to integrate the series term by term.
Because of orthogonality we have
\begin{equation*}
\begin{split}
\langle f , y_m \rangle & =
\int_a^b f(x) \, y_m (x) \, r(x) ~ dx\\
&= \sum_{n=1}^\infty c_n \int_a^b y_n(x) \, y_m (x) \, r(x) ~ dx\\
&= c_m \int_a^b y_m(x) \, y_m (x) \, r(x) ~ dx = c_m \langle y_m , y_m \rangle
.
\end{split}
\end{equation*}
Hence,
\begin{equation} \label{sl:cm}
\boxed{~~
c_m = \frac{\langle f , y_m \rangle}{\langle y_m , y_m \rangle}
=
\frac{\int_a^b f(x) \, y_m (x)\, r(x) ~ dx}%
{\int_a^b {\bigl(y_m(x)\bigr)}^2 \, r(x) ~dx} .
~~}
\end{equation}

Note that $y_m$ are known up to a constant multiple, so we could have picked
a scalar multiple of an eigenfunction such that
$\langle y_m , y_m \rangle = 1$ (if we had an arbitrary eigenfunction
$\tilde{y}_m$, divide it
by $\sqrt{\langle \tilde{y}_m , \tilde{y}_m \rangle}$).
When
$\langle y_m , y_m \rangle = 1$
we have the
simpler form $c_m = \langle f, y_m \rangle$ as we did for the
Fourier series.  The following theorem holds
more generally, but the statement given is enough for our purposes.

\begin{theorem}
Suppose $f$ is a piecewise smooth continuous function on $[a,b]$.  If $y_1,
y_2, \ldots$ are eigenfunctions of a regular Sturm-Liouville problem,
one for each eigenvalue,
then there exist real constants $c_1, c_2, \ldots$ given by \eqref{sl:cm}
such that
\eqref{sl:fdecomp} converges and holds for $a < x < b$.
\end{theorem}

\begin{example}
Take the simple Sturm-Liouville problem
\begin{align*}
& y'' + \lambda y = 0, \quad 0 < x < \frac{\pi}{2} , \\
& y(0) =0, \quad y'\left(\frac{\pi}{2}\right) = 0 .
\end{align*}
The above is a regular problem and furthermore we know by
\thmvref{sl:slregthm}
that $\lambda \geq 0$.

Suppose $\lambda = 0$, then the general solution is $y(x) = Ax + B$,
we plug in the
initial conditions to get $0=y(0) = B$, and $0 = y'(\frac{\pi}{2}) = A$, hence
$\lambda = 0$ is not an eigenvalue.
The general solution, therefore, is
\begin{equation*}
y(x) = A \cos (\! \sqrt{\lambda} \, x ) + B \sin (\! \sqrt{\lambda} \, x) .
\end{equation*}
Plugging in the boundary conditions we get
$0 = y(0) = A$ and $0 = y'\bigl(\frac{\pi}{2}\bigr)
= \sqrt{\lambda} \, B \cos \bigl(\!\sqrt{\lambda} \, \frac{\pi}{2}\bigr)$.
$B$ cannot be zero and hence $\cos \bigl(\! \sqrt{\lambda} \,
\frac{\pi}{2}\bigr) = 0$.
This means that
$\sqrt{\lambda} \,\frac{\pi}{2}$ must be an odd integral multiple of
$\frac{\pi}{2}$,
i.e.\ $(2n-1)\frac{\pi}{2} = \sqrt{\lambda_n} \,\frac{\pi}{2}$.
Hence
\begin{equation*}
\lambda_n = {(2n-1)}^2 .
\end{equation*}
We can take $B = 1$.  Hence our eigenfunctions are
\begin{equation*}
y_n(x) = \sin \bigl( (2n-1)x \bigr) .
\end{equation*}
Finally we compute
\begin{equation*}
\int_0^{\frac{\pi}{2}} {\Bigl( \sin \bigl( (2n-1)x \bigr) \Bigr)}^2 ~ dx
= \frac{\pi}{4} .
\end{equation*}

So any piecewise smooth function on $[0,\frac{\pi}{2}]$ can be written as
\begin{equation*}
f(x) = \sum_{n=1}^\infty c_n \sin \bigl( (2n-1)x \bigr) ,
\end{equation*}
where
\begin{equation*}
c_n = \frac{\langle f , y_n \rangle}{\langle y_n , y_n \rangle}
= \frac{\int_0^{\frac{\pi}{2}} f(x) \, \sin \bigl( (2n-1)x \bigr) ~ dx
}{\int_0^{\frac{\pi}{2}} {\Bigl(\sin \bigl((2n-1)x\bigr)\Bigr)}^2 ~ dx}
= \frac{4}{\pi} \int_0^{\frac{\pi}{2}} f(x) \,\sin \bigl( (2n-1)x \bigr) ~ dx .
\end{equation*}

Note that the series converges to an odd $2\pi$-periodic (not
$\pi$-periodic!) extension of $f(x)$.
\end{example}

\begin{exercise}[challenging]
In the above example, the function is defined on $0 < x < \frac{\pi}{2}$,
yet the series converges to an odd $2\pi$-periodic extension of $f(x)$.
Find out how is the extension defined for $\frac{\pi}{2} < x < \pi$.
\end{exercise}

\subsection{Exercises}

\begin{exercise}
Find eigenvalues and eigenfunctions of
\begin{equation*}
y''+\lambda y = 0, \quad y(0)- y'(0) = 0, \quad y(1) = 0 .
\end{equation*}
\end{exercise}

\begin{exercise}
Expand the function $f(x) = x$ on $0 \leq x \leq 1$ using eigenfunctions
of the system
\begin{equation*}
y'' + \lambda y = 0, \quad y'(0) = 0, \quad y(1) = 0 .
\end{equation*}
\end{exercise}

\begin{exercise}
Suppose that you had a Sturm-Liouville problem on the interval
$[0,1]$ and came up with
$y_n(x) = \sin (\gamma n x)$, where $\gamma > 0$ is some constant.
Decompose $f(x) = x$, $0 < x < 1$ in terms of these eigenfunctions.
\end{exercise}

\begin{exercise}
Find eigenvalues and eigenfunctions of
\begin{equation*}
y^{(4)}+\lambda y = 0, \quad y(0) = 0, \quad y'(0) = 0, \quad y(1) = 0, \quad
y'(1) = 0 .
\end{equation*}
This problem is not a Sturm-Liouville problem, but the idea is the same.
\end{exercise}

\begin{exercise}[more challenging]
Find eigenvalues and eigenfunctions for
\begin{equation*}
\frac{d}{dx} (e^x y') + \lambda e^x y = 0, \quad y(0) = 0, \quad y(1) = 0 .
\end{equation*}
%Or at least set up the integrals you would need to solve.
Hint: First write the system as a constant coefficient system to find
general solutions.  Do note that \thmvref{sl:slregthm} guarantees $\lambda \geq 0$.
\end{exercise}

\setcounter{exercise}{100}

\begin{exercise}
Find eigenvalues and eigenfunctions of
\begin{equation*}
y'' + \lambda y = 0, \quad y(-1) = 0, \quad y(1) = 0 .
\end{equation*}
\end{exercise}
\exsol{%
$\lambda_n = \frac{(2n-1)\pi}{2}$, $n=1,2,3,\ldots$, 
$y_n = \cos\left(\frac{(2n-1)\pi}{2} x\right)$
}

\begin{exercise}
Put the following problems into the standard form for Sturm-Liouville
problems, that is, find $p(x)$, $q(x)$, $r(x)$,
$\alpha_1$,
$\alpha_2$,
$\beta_1$, and
$\beta_2$,
and decide if the problems are regular or not.\\
a) $x y'' + \lambda y = 0$ for $0 < x < 1$, $y(0) = 0$, $y(1) = 0$,\\
b)\footnote{In a previous version of the book, a typo rendered the equation
as $(1+x^2) y'' - 2xy' + (\lambda-x^2) y = 0$ ending up with something
harder than intended.  Try this equation for a further challenge.}
$(1+x^2) y'' + 2xy' + (\lambda-x^2) y = 0$ for $-1 < x < 1$, $y(-1) = 0$, $y(1)+y'(1) =
0$.
\end{exercise}
\exsol{%
a)~$p(x) = 1$, $q(x) = 0$, $r(x) = \frac{1}{x}$, $\alpha_1 = 1$, $\alpha_2 =
0$, $\beta_1 = 1$, $\beta_2 = 0$.  The problem is not regular.
b)~$p(x) = 1+x^2$, $q(x) = x^2$, $r(x) = 1$, $\alpha_1 = 1$, $\alpha_2 =
0$, $\beta_1 = 1$, $\beta_2 = 1$.  The problem is regular.
}

%%%%%%%%%%%%%%%%%%%%%%%%%%%%%%%%%%%%%%%%%%%%%%%%%%%%%%%%%%%%%%%%%%%%%%%%%%%%%%

\sectionnewpage
\section{Application of eigenfunction series}
\label{sec:appeig}

\sectionnotes{1 lecture\EPref{, \S10.2 in \cite{EP}}\BDref{,
exercises in \S11.2 in \cite{BD}}}

The eigenfunction series can arise even from higher order equations.
Suppose we have an elastic beam (say made of steel).  We will study the
transversal vibrations of the beam.  That is, assume the beam lies along
the $x$ axis and let $y(x,t)$ measure the displacement of the point $x$
on the beam at time $t$.  See \figurevref{appeig:transbeamfig}.

\begin{figure}[h!t]
\capstart
\begin{center}
\inputpdft{trans-beam}
\caption{Transversal vibrations of a beam.\label{appeig:transbeamfig}}
\end{center}
\end{figure}

The equation that governs this setup is
\begin{equation*}
a^4 \frac{\partial^4 y}{\partial x^4} + \frac{\partial^2 y}{\partial t^2} = 0,
\end{equation*}
for some constant $a > 0$.%\EPref{ ($a^4 = \nicefrac{EI}{\rho}$ in \cite{EP})}.

Suppose the beam is of length 1 simply supported (hinged) at the ends.
The beam is displaced by some function $f(x)$ at time $t=0$ and then
let go (initial velocity is 0).  Then $y$ satisfies:
\begin{equation} \label{appeig:beameq}
\begin{aligned}
& a^4 y_{xxxx} + y_{tt} = 0 \quad (0 < x < 1, t > 0), \\
& y(0,t) = y_{xx}(0,t) = 0 , \\
& y(1,t) = y_{xx}(1,t) = 0 , \\
& y(x,0) = f(x), \quad y_{t}(x,0) = 0 .
\end{aligned}
\end{equation}

Again we try $y(x,t) = X(x)T(t)$ and plug in to get
$a^4 X^{(4)}T + XT'' = 0$ or 
\begin{equation*}
\frac{X^{(4)}}{X} = \frac{- T''}{a^4T} = \lambda .
\end{equation*}
We note that we want $T'' + \lambda a^4T = 0$.  Let us assume that $\lambda >
0$.  We can argue that we expect vibration and not exponential growth nor
decay in the $t$ direction (there is no friction in our model for instance).
Similarly $\lambda = 0$ will not occur.

\begin{exercise}
Try to justify $\lambda > 0$ just from the equations.
\end{exercise}

Write $\omega^4 = \lambda$, so that we do not need to write the fourth root
all the time.  For $X$ we get the equation $X^{(4)} - \omega^4 X = 0$.  The
general solution is
\begin{equation*}
X(x) = A e^{\omega x} + B e^{-\omega x} + C \sin (\omega x) +
D \cos (\omega x) .
\end{equation*}
Now $0 = X(0) = A+B+D$, $0 = X''(0) = \omega^2 (A + B - D)$.  Hence, $D = 0$ and $A+B = 0$, or $B = - A$.  So we have
\begin{equation*}
X(x) = A e^{\omega x} - A e^{-\omega x} + C \sin (\omega x) .
\end{equation*}
Also $0 = X(1) = A (e^{\omega} - e^{-\omega}) + C \sin \omega$, and
$0 = X''(1) = A \omega^2 (e^{\omega} - e^{-\omega}) - C \omega^2 \sin \omega$.
This means that $C \sin \omega  = 0$ and 
$A (e^{\omega} - e^{-\omega}) = 2 A \sinh \omega = 0$.  If $\omega > 0$, then
$\sinh \omega \not= 0$ and so $A = 0$.  This means that $C \not=0$ otherwise
$\lambda$ is not an
eigenvalue.  Also $\omega$ must be an integer multiple of
$\pi$.   Hence $\omega = n \pi$ and $n \geq 1$ (as $\omega > 0$).  We can take
$C=1$.  So the eigenvalues are $\lambda_n = n^4 \pi^4$ and corresponding eigenfunctions
are $\sin (n \pi x)$.

Now 
$T'' + n^4 \pi^4 a^4 T = 0$.  The general solution is $T(t) =
A \sin (n^2 \pi^2 a^2 t) + B \cos (n^2 \pi^2 a^2 t)$.  But $T'(0) = 0$ and hence
we must have $A=0$ and we can take $B=1$ to make $T(0) = 1$ for convenience.
So our solutions are $T_n(t) = \cos (n^2 \pi^2 a^2 t)$.

As eigenfunctions are just sines again, we can decompose the function
$f(x)$ on $0 < x < 1$ using the sine series.
We find numbers $b_n$ such that for
$0 < x < 1$ we have
\begin{equation*}
f(x) = \sum_{n=1}^\infty b_n \sin (n \pi x) .
\end{equation*}
Then the solution to \eqref{appeig:beameq} is
\begin{equation*}
y(x,t) = \sum_{n=1}^\infty b_n
X_n(x) T_n(t)
= \sum_{n=1}^\infty b_n
\sin (n \pi x)  \cos ( n^2 \pi^2 a^2 t ) .
\end{equation*}
The point is that $X_nT_n$ is a solution that satisfies all the homogeneous
conditions (that is, all conditions except the initial position).  And since
and $T_n(0) = 1$, we have
\begin{equation*}
y(x,0) = \sum_{n=1}^\infty b_n X_n(x) T_n(0) = 
\sum_{n=1}^\infty b_n X_n(x) =
\sum_{n=1}^\infty b_n
\sin (n \pi x) = f(x) .
\end{equation*}
So $y(x,t)$ solves \eqref{appeig:beameq}.

The natural (angular) frequencies of the system are $n^2 \pi^2 a^2$.
These frequencies are all integer multiples of the fundamental frequency
$\pi^2 a^2$, so we get a nice musical note.  The exact frequencies
and their amplitude
are what musicians call the \emph{\myindex{timbre}} of the note (outside
of music it is called the spectrum).

The timbre of a beam
is different than for a vibrating string where we get \myquote{more}
of the lower frequencies since we get all integer multiples,
$1,2,3,4,5,\ldots$.  For a steel beam we get
only the square multiples $1,4,9,16,25,\ldots$.  That is why when you hit a
steel beam you hear a very pure sound.  The sound of a
xylophone or vibraphone is, therefore, very different from a guitar or piano.

\begin{example}
Let us assume that $f(x) = \frac{x(x-1)}{10}$.  
On $0 < x < 1$ we have (you know how to do this by now)
\begin{equation*}
f(x) = \sum_{\substack{n=1\\n \text{~odd}}}^\infty \frac{4}{5\pi^3 n^3}
\sin (n \pi x) .
\end{equation*}
Hence, the solution to \eqref{appeig:beameq} with the given initial
position $f(x)$ is
\begin{equation*}
y(x,t) = \sum_{\substack{n=1\\n \text{~odd}}}^\infty \frac{4}{5\pi^3 n^3}
\sin (n \pi x) \cos ( n^2 \pi^2 a^2 t ) .
\end{equation*}
\end{example}

\subsection{Exercises}

\begin{exercise}
Suppose you have a beam of length 5 with free ends.  Let $y$ be the
transverse deviation of the beam at position $x$ on the beam ($0 < x < 5$).
You know that the
constants are such that this satisfies the equation $y_{tt} + 4 y_{xxxx} =
0$.   Suppose you know that the initial shape of the beam is the graph of
$x(5-x)$, and the initial velocity is uniformly equal to 2 (same for each $x$)
in the positive $y$ direction.  Set up the equation together with the
boundary and initial conditions.  Just set up, do not solve.
\end{exercise}

\begin{exercise}
Suppose you have a beam of length 5 with one end free and one end fixed
(the fixed end is at $x=5$).
Let $u$ be the
longitudinal deviation of the beam at position $x$ on the beam ($0 < x < 5$).
You know that the
constants are such that this satisfies the equation $u_{tt} = 4 u_{xx}$.
Suppose you know that the initial displacement of the beam
is $\frac{x-5}{50}$, and the initial velocity is $\frac{-(x-5)}{100}$
in the positive $u$ direction.  Set up the equation together with the
boundary and initial conditions.  Just set up, do not solve.
\end{exercise}

\begin{exercise}
Suppose the beam is $L$ units long, everything else kept the same
as in \eqref{appeig:beameq}.  What is the equation and the series
solution?
\end{exercise}

\begin{exercise}
Suppose you have 
\begin{equation*}
\begin{aligned}
& a^4 y_{xxxx} + y_{tt} = 0 \quad (0 < x < 1, t > 0) , \\
& y(0,t) = y_{xx}(0,t) = 0,\\
& y(1,t) = y_{xx}(1,t) = 0 ,\\
& y(x,0) = f(x), \quad y_{t}(x,0) = g(x) .
\end{aligned}
\end{equation*}
That is, you have also an initial velocity.  Find a series solution.  Hint:
Use the same idea as we did for the wave equation.
\end{exercise}

\setcounter{exercise}{100}

\begin{exercise}
Suppose you have a beam of length 1 with hinged ends.  Let $y$ be the
transverse deviation of the beam at position $x$ on the beam ($0 < x < 1$).
You know that the
constants are such that this satisfies the equation $y_{tt} + 4 y_{xxxx} =
0$.   Suppose you know that the initial shape of the beam is the graph of
$\sin (\pi x)$, and the initial velocity is 0.  Solve for $y$.
\end{exercise}
\exsol{%
$y(x,t) = \sin(\pi x) \cos (4 \pi^2 t)$
}

\begin{exercise}
Suppose you have a beam of length 10 with two fixed ends.  Let $y$ be the
transverse deviation of the beam at position $x$ on the beam ($0 < x < 10$).
You know that the
constants are such that this satisfies the equation $y_{tt} + 9 y_{xxxx} =
0$.   Suppose you know that the initial shape of the beam is the graph of
$\sin(\pi x)$, and the initial velocity is uniformly equal to $x(10-x)$.
Set up the equation together with the
boundary and initial conditions.  Just set up, do not solve.
\end{exercise}
\exsol{%
$9 y_{xxxx} + y_{tt} = 0 \quad (0 < x < 10, t > 0)$, \quad
$y(0,t) = y_{x}(0,t) = 0$, \quad
$y(10,t) = y_{x}(10,t) = 0$, \quad
$y(x,0) = \sin(\pi x), \quad y_{t}(x,0) = x(10-x)$.
}

%%%%%%%%%%%%%%%%%%%%%%%%%%%%%%%%%%%%%%%%%%%%%%%%%%%%%%%%%%%%%%%%%%%%%%%%%%%%%%

\sectionnewpage
\section{Steady periodic solutions}
\label{sps:section}

\sectionnotes{1--2 lectures\EPref{, \S10.3 in \cite{EP}}\BDref{,
not in \cite{BD}}}

\subsection{Forced vibrating string}

Suppose that we have a guitar string of length $L$.  We have studied the
wave equation problem in this case, where $x$ was the position on the string,
$t$ was time and $y$ was the displacement of the string.  See
\figurevref{sps:vibstrfig}.

\begin{figure}[h!t]
\capstart
\begin{center}
\inputpdft{sps-vibstr}
\caption{Vibrating string.\label{sps:vibstrfig}}
\end{center}
\end{figure}

The problem is governed by the equations
\begin{equation} \label{sps:freevib}
\begin{array}{ll}
y_{tt} = a^2 y_{xx} , & \\
\noalign{\smallskip}
y(0,t) = 0 , & y(L,t) = 0 , \\
y(x,0) = f(x) , & y_t(x,0) = g(x) .
\end{array}
\end{equation}
We saw previously that 
the solution is of the form
\begin{equation*}
y = 
\sum_{n=1}^\infty \left( A_n \cos \left( \frac{n\pi a}{L} \, t \right) +
B_n \sin \left( \frac{n\pi a}{L} \, t \right) \right)
\sin \left( \frac{n\pi}{L} \, x \right) ,
\end{equation*}
where $A_n$ and $B_n$ were determined by the initial conditions.  The natural
frequencies of the system are the (angular) frequencies $\frac{n \pi a}{L}$
for integers $n \geq 1$.

But these are free vibrations.  What if there is an external force acting on
the string.  Let us assume say air vibrations (noise), for example a second
string.  Or perhaps a jet engine.  For simplicity, assume nice pure
sound and assume the force is uniform at every position on the string.
Let us say $F(t) = F_0 \cos (\omega t)$ as force per unit mass.  Then our wave
equation becomes (remember force is mass times acceleration)
\begin{equation} \label{sps:forcedeq}
y_{tt} = a^2 y_{xx} + F_0 \cos ( \omega t) ,
\end{equation}
with the same boundary conditions of course.

We want to find the solution here that satisfies the above equation and
\begin{equation} \label{sps:forcedinitcond}
y(0,t) = 0, \qquad y(L,t) = 0, \qquad
y(x,0) = 0, \qquad y_t(x,0) = 0.
\end{equation}
That is, the string is initially at rest.  First we find a particular
solution $y_p$ of \eqref{sps:forcedeq} that satisfies
$y(0,t) = y(L,t) = 0$.  We define the functions $f$ and $g$ as
\begin{equation*}
f(x) = -y_p(x,0), \qquad g(x) = -\frac{\partial y_p}{\partial t} (x,0) .
\end{equation*}
We then find solution $y_c$ of \eqref{sps:freevib}.  If we add the two
solutions, we find that $y = y_c + y_p$ solves \eqref{sps:forcedeq} with
the initial conditions.

\begin{exercise}
Check that $y = y_c + y_p$ solves \eqref{sps:forcedeq} and the
side conditions \eqref{sps:forcedinitcond}.
\end{exercise}

So the big issue here is to find the particular solution $y_p$.
We look at the equation and we make an educated guess
\begin{equation*}
y_p(x,t) = X(x) \cos (\omega t) .
\end{equation*}
We plug in to get
\begin{equation*}
-\omega^2 X \cos ( \omega t) = a^2 X'' \cos ( \omega t) +
F_0 \cos ( \omega t ) ,
\end{equation*}
or
$-\omega^2 X = a^2 X'' + F_0$ after canceling the cosine.
We know how to find a general solution to this equation (it is a
nonhomogeneous constant coefficient equation).
The general solution is
\begin{equation*}
X(x) = A \cos \left( \frac{\omega}{a} \, x \right)
+ B \sin \left( \frac{\omega}{a} \, x \right) -
\frac{F_0}{\omega^2} .
\end{equation*}
The endpoint conditions imply $X(0) = X(L) = 0$.  So
\begin{equation*}
0 = X(0) = A - \frac{F_0}{\omega^2} ,
\end{equation*}
or $A = \frac{F_0}{\omega^2}$, and also
\begin{equation*}
0 = X(L)
= \frac{F_0}{\omega^2} \cos \left( \frac{\omega L}{a} \right)
+ B \sin \left( \frac{\omega L}{a} \right) -
\frac{F_0}{\omega^2} .
\end{equation*}
Assuming that $\sin ( \frac{\omega L}{a} )$ is not zero we can solve for $B$ to
get
\begin{equation} \label{natfreq:Beq}
B = 
\frac{-F_0 \left( \cos \left( \frac{\omega L}{a} \right) - 1 \right)}%
{\omega^2 \sin \left( \frac{\omega L}{a} \right)}.
\end{equation}
Therefore,
\begin{equation*}
X(x) =
\frac{F_0}{\omega^2} \left(
\cos \left(  \frac{\omega}{a} \, x \right) -
\frac{\cos \left( \frac{\omega L}{a} \right) - 1}%
{\sin \left( \frac{\omega L}{a} \right)}
\sin \left( \frac{\omega}{a} \, x \right)
- 1
\right) .
\end{equation*}
The particular solution $y_p$ we are looking for is
\begin{equation*}
\boxed{~~
y_p(x,t) =
\frac{F_0}{\omega^2} \left(
\cos \left( \frac{\omega}{a} \, x \right) -
\frac{\cos \left( \frac{\omega L}{a} \right) - 1}%
{\sin \left( \frac{\omega L}{a}\right)}
\sin \left( \frac{\omega}{a} \, x \right)
-1
\right)
\cos ( \omega t) .
~~}
\end{equation*}

\begin{exercise}
Check that $y_p$ works.
\end{exercise}

Now we get to the point that we skipped.  Suppose 
$\sin ( \frac{\omega L}{a} ) = 0$.  What this means is that
$\omega$ is equal to one of the natural frequencies of the system,
i.e.\ a multiple of $\frac{\pi a}{L}$.  We notice that if $\omega$
is not equal to a multiple of the base frequency, but is very close,
then the coefficient $B$ in \eqref{natfreq:Beq} seems to
become very large.  But let us not jump to conclusions just yet.
When $\omega = \frac{n \pi a}{L}$
for $n$ even, then $\cos (\frac{\omega L}{a}) = 1$ and hence we really get that
$B=0$.  So resonance occurs only when 
both $\cos (\frac{\omega L}{a}) = -1$ and
$\sin (\frac{\omega L}{a}) = 0$.  That is when $\omega = \frac{n \pi a }{L}$
for \emph{odd} $n$.

We could again solve for the resonance solution if we wanted to, but it is, in the right sense, the limit of the solutions as $\omega$ gets
close to a resonance frequency.
In real life, pure resonance never occurs anyway.
%You would get a particular solution that would have a term
%such as $t \cos (\omega t)$ in it (though it would be harder (but possible)
%to find the solution using the above method).

The above calculation explains why a string will begin to vibrate if the
identical string is plucked close by.  In the absence of friction this vibration
would get louder and louder as time goes on.
On the other hand, you are unlikely to get large vibration if the forcing 
frequency is not close to a resonance frequency even if you have a jet engine
running close to
the string.  That is, the amplitude will not keep
increasing unless you tune to just the right frequency.

Similar resonance phenomena occur when you break a wine glass using human
voice (yes
this is possible, but not easy%
\footnote{\emph{Mythbusters}, episode 31, Discovery Channel, originally aired
may 18th 2005.}) if you happen to hit just the right
frequency.  Remember a glass has much purer sound, i.e.\ it is more like a
vibraphone, so there are far fewer resonance frequencies to hit.

When the forcing function is more complicated, you decompose it in terms of
the Fourier series and apply the above result.  You may also need to solve
the above problem if the forcing function is a sine rather than a cosine,
but if you think about it, the solution is almost the same.
%  That is, the
%particular solution has $\sin$ instead of cosine and otherwise is identical.
%Your complementary solution will be slightly different as $f$ and $g$
%will differ.

\begin{example}
Let us do the computation for specific values.
Suppose $F_0 = 1$ and $\omega = 1$ and $L=1$ and $a=1$.  Then 
\begin{equation*}
y_p(x,t) =
\left(
\cos (x) -
\frac{\cos (1) - 1}{\sin (1)}
\sin (x)
-1
\right)
\cos (t) .
\end{equation*}
Write $B = \frac{\cos (1) - 1}{\sin (1)}$ for simplicity.

Then plug in $t=0$ to get
\begin{equation*}
f(x) =- y_p(x,0) = 
- \cos x +
B \sin x
+1 ,
\end{equation*}
and after differentiating in $t$ we see that 
$g(x) = -\frac{\partial y_p}{\partial t}(x,0) = 0$.

Hence to find $y_c$ we need to solve the problem
\begin{align*}
& y_{tt} = y_{xx} , \\
& y(0,t) = 0 , \quad y(1,t) = 0 , \\
& y(x,0) = - \cos x + B \sin x +1 , \\
& y_t(x,0) = 0 .
\end{align*}
Note that the formula that we use to define $y(x,0)$ is not odd,
hence it is not a simple matter of plugging in to apply the D'Alembert
formula directly!  You must define $F$ to be the odd, 2-periodic
extension of $y(x,0)$.  Then our solution would look like
\begin{equation} \label{natfreq:exsol}
y(x,t) = 
\frac{F(x+t) + F(x-t)}{2} + 
\left(
\cos (x) -
\frac{\cos (1) - 1}{\sin (1)}
\sin (x)
-1
\right)
\cos (t) .
\end{equation}

\begin{figure}[h!t]
\capstart
\begin{center}
\diffyincludegraphics{width=5in}{width=7.5in}{natfreq-forcedvib}
\caption[Plot of $y(x,t)$.]{Plot of $y(x,t) = \frac{F(x+t) + F(x-t)}{2} + \left( \cos (x) -
\frac{\cos (1) - 1}{\sin (1)} \sin (x) -1 \right) \cos (t)$.%
\label{natfreq:forcedvibfig}}
%%%
%%% Tex4ht is giving trouble here because of the math, it for some
%%% reason does not like the left and right.  Giving the optional short
%%% caption without it makes the problem go away.
%%%
\end{center}
\end{figure}

It is not hard to compute specific values
for an odd periodic extension of a function and
hence \eqref{natfreq:exsol} is a wonderful solution to the problem.
For example it is very easy to have a computer do it, unlike a series solution.
A plot is given in \figurevref{natfreq:forcedvibfig}.
\end{example}

\subsection{Underground temperature oscillations}

Let $u(x,t)$ be the temperature at a certain location at depth $x$
underground at time $t$.  See \figurevref{sps:groundtempfig}.

%\begin{figure}[h!t]
\begin{diffyfloatingfigurer}{2.5in}{2.5in}
\capstart
\begin{center}
\inputpdft{sps-groundtemp}
\caption{Underground temperature.\label{sps:groundtempfig}}
\end{center}
\end{diffyfloatingfigurer}
%\end{figure}

The temperature $u$ satisfies the heat equation $u_t = ku_{xx}$, where $k$
is the diffusivity of the soil.
We know the temperature at the surface $u(0,t)$ from weather
records.  Let us assume for simplicity that
\begin{equation*}
u(0,t) = T_0 + A_0 \cos (\omega t) ,
\end{equation*}
where $T_0$ is the yearly mean
temperature, and
$t=0$ is midsummer (you can put
negative sign above to make it midwinter if you wish).  $A_0$ gives 
the typical variation for the year.  That is,
the hottest temperature is $T_0 + A_0$ and the coldest is $T_0 - A_0$.
For simplicity, we will assume that $T_0 = 0$.
The frequency $\omega$ is picked depending on the units of $t$, such that
when $t=\unit[1]{year}$, then $\omega t = 2 \pi$.  For example if $t$ is
in years, then $\omega = 2\pi$.

It seems reasonable that the temperature at depth $x$ will also oscillate
with the same frequency.  This, in fact, will be the steady periodic
solution, independent of the initial conditions.
So we are looking for a solution of the form
\begin{equation*}
u(x,t) = V(x) \cos (\omega t) + W (x) \sin ( \omega t) ,
\end{equation*}
for the problem
\begin{equation} \label{sps:ueq}
u_t = k u_{xx}, \qquad u(0,t) = A_0 \cos ( \omega t) .
\end{equation}

We will employ the complex exponential here to make calculations simpler.
Suppose we have a complex valued function
\begin{equation*}
h(x,t) = X(x)\, e^{i\omega t} .
\end{equation*}
We will look for an $h$ such that $\operatorname{Re} h = u$.
To find an $h$, whose real part satisfies \eqref{sps:ueq}, we look for
an $h$ such that
\begin{equation} \label{sps:heq}
h_t = k h_{xx}, \qquad h(0,t) = A_0 e^{i\omega t} .
\end{equation}

\begin{exercise}
Suppose $h$ satisfies \eqref{sps:heq}.
Use \hyperref[eulersformula]{Euler's formula} for the complex exponential to
check that $u = \operatorname{Re} h$ satisfies \eqref{sps:ueq}.
\end{exercise}

Substitute $h$ into \eqref{sps:heq}.
\begin{equation*}
i\omega X e^{i\omega t} = k X'' e^{i \omega t} .
\end{equation*}
Hence,
\begin{equation*}
k X''  - i \omega X = 0 ,
\end{equation*}
or 
\begin{equation*}
X''  - \alpha^2 X = 0 ,
\end{equation*}
where $\alpha = \pm \sqrt{\frac{i\omega}{k}}$.  Note that $\pm \sqrt{i} = \pm
\frac{1+i}{\sqrt{2}}$ so you could simplify to
$\alpha = \pm (1+i)\sqrt{\frac{\omega}{2k}}$.
Hence the general solution is
\begin{equation*}
X(x) = A e^{-(1+i)\sqrt{\frac{\omega}{2k}} \, x}
+ B e^{(1+i)\sqrt{\frac{\omega}{2k}} \, x} .
\end{equation*}
We assume that an $X(x)$ that solves the problem must be bounded as $x \to
\infty$ since $u(x,t)$ should be bounded (we are not worrying about the earth
core!).
If you use \hyperref[eulersformula]{Euler's formula} to expand the complex exponentials, you will
note that the second term will be unbounded (if $B \not = 0$),
while the first term is always bounded.  Hence $B=0$.

\begin{exercise}
Use \hyperref[eulersformula]{Euler's formula} to show that
$e^{(1+i)\sqrt{\frac{\omega}{2k}} \, x}$ is unbounded as $x \to \infty$,
while $e^{-(1+i)\sqrt{\frac{\omega}{2k}} \, x}$ is bounded
as $x \to \infty$.
\end{exercise}

Furthermore, $X(0) = A_0$ since $h(0,t) = A_0 e^{i \omega t}$.
Thus $A=A_0$.  This means that
\begin{equation*}
h(x,t) = A_0 e^{-(1+i)\sqrt{\frac{\omega}{2k}} \, x} e^{i \omega t}
=
A_0 e^{-(1+i)\sqrt{\frac{\omega}{2k}} \, x + i \omega t}
=
A_0 e^{-\sqrt{\frac{\omega}{2k}} \, x}
e^{i(\omega t - \sqrt{\frac{\omega}{2k}} \, x)} .
\end{equation*}
We will need to get the real part of $h$, so
we apply \hyperref[eulersformula]{Euler's formula} to get
\begin{equation*}
h(x,t) =
A_0 e^{-\sqrt{\frac{\omega}{2k}} \, x}
\left(\cos \left(\omega t - \sqrt{\frac{\omega}{2k}}\, x\right) + 
i \sin \left(\omega t - \sqrt{\frac{\omega}{2k}}\, x\right) \right) .
\end{equation*}
Then finally
\begin{equation*}
u(x,t) = \operatorname{Re} h(x,t) =
A_0 e^{-\sqrt{\frac{\omega}{2k}}\, x}
\cos \left(\omega t - \sqrt{\frac{\omega}{2k}}\, x\right) .
\end{equation*}
Yay!

Notice the phase is different at different depths.  At depth $x$ the
phase is delayed by $x \sqrt{\frac{\omega}{2k}}$.
For example in cgs units (centimeters-grams-seconds)\index{cgs units}
we have $k=0.005$ (typical value for soil),
$\omega = \frac{2\pi}{\text{seconds in a year}}
= \frac{2\pi}{31,557,341} \approx 1.99 \times {10}^{-7}$.   Then
if we compute where the phase shift $x \sqrt{\frac{\omega}{2k}} = \pi$
we find the depth in centimeters where the seasons are reversed.  That is,
we get the depth at which summer is the coldest and winter is the warmest.
We get
approximately 700 centimeters, which is approximately 23 feet below ground.

Be careful not to jump to conclusions.  The temperature swings decay rapidly as you dig deeper.  The
amplitude of the temperature swings is
$A_0 e^{-\sqrt{\frac{\omega}{2k}} x}$.  This function decays \emph{very}
quickly as $x$ (the depth) grows.
Let us again take
typical parameters as above.  We will also assume that
our surface temperature swing is $\pm {15}^\circ$ Celsius, that is,
$A_0 = 15$.  Then the maximum temperature variation at 700 centimeters
is only $\pm {0.66}^\circ$ Celsius.

You need not dig very deep to get an effective
\myquote{refrigerator,} with nearly constant temperature.  That is why wines are kept in a cellar; you need consistent
temperature.
The temperature differential could also be used for energy.  A home could
be heated or cooled by taking advantage of the above fact.
Even without the earth core you could heat a home in the winter and cool it
in the summer.  The earth core makes the
temperature higher the deeper you dig, although you need to dig somewhat
deep to feel a difference.
We did not take that into account above.

\subsection{Exercises}

\begin{exercise} \label{sps:sinforceexr}
Suppose that the forcing function for the vibrating string
is $F_0 \sin (\omega t)$.  Derive the particular solution $y_p$.
\end{exercise}

\begin{exercise}
Take the forced vibrating string.
Suppose that $L=1$, $a=1$.  Suppose that the forcing function
is the square wave that is 1 on the interval $0 < x < 1$ and
$-1$ on the interval $-1 < x< 0$.
Find the particular solution.  Hint: You may want to use result
of \exerciseref{sps:sinforceexr}.
\end{exercise}

\begin{exercise}
The units are cgs (centimeters-grams-seconds)\index{cgs units}.
For $k=0.005$, $\omega = 1.991 \times {10}^{-7}$, $A_0 = 20$.
Find the depth at which the temperature variation is half ($\pm 10$
degrees) of what it is on the surface.
\end{exercise}

\begin{exercise}
Derive the solution for underground temperature oscillation without assuming
that $T_0 = 0$.
\end{exercise}

\setcounter{exercise}{100}

\begin{exercise}
Take the forced vibrating string.
Suppose that $L=1$, $a=1$.  Suppose that the forcing function
is a sawtooth, that is $\lvert x \rvert -\frac{1}{2}$
on $-1 < x < 1$ extended periodically.
Find the particular solution.
\end{exercise}
\exsol{%
$y_p(x,t) =
\sum\limits_{\substack{n=1 \\ n \text{ odd}}}^\infty
\frac{-4}{n^4 \pi^4} \,
\left(
\cos(n \pi x ) -
\frac{\cos ( n \pi ) - 1}%
{\sin( n \pi)}
\sin( n \pi x)
-1
\right)
\cos (n \pi t) .
$
}

\begin{exercise}
The units are cgs (centimeters-grams-seconds)\index{cgs units}.
For $k=0.01$, $\omega = 1.991 \times {10}^{-7}$, $A_0 = 25$.
Find the depth at which the summer is again the hottest point.
\end{exercise}
\exsol{%
%$x \sqrt{\frac{\omega}{2k}} = 2\pi$
%$x = 2\pi\sqrt{\frac{2k}{\omega}}$
%$x = 2\pi\sqrt{\frac{0.02}{1.991 x 10^-7}}$
Approximately 1991 centimeters
}
